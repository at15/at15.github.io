\documentclass[11pt, letterpaper]{simple-cv}
% https://github.com/at15/at15.github.io/issues/10 to solve LaTeX Font Warning: Font shape `TU/ptm/m/n' undefined
% https://tex.stackexchange.com/questions/358261/latex-font-warning-after-updating-to-texlive-2016/
\usepackage[OT1]{fontenc} 
% to use times new roman https://tex.stackexchange.com/questions/153168/how-to-set-document-font-to-times-new-roman-by-command
\usepackage{mathptmx}
\usepackage{enumitem} % https://tex.stackexchange.com/questions/10684/vertical-space-in-lists
\setlist{noitemsep}
% \setlist{nosep}
\setlist[itemize]{leftmargin=*} % https://tex.stackexchange.com/questions/91124/itemize-removing-natural-indent

\begin{document}

\name{Pinglei Guo}
% TODO: put into two lines, address + phone, and the rest
\centerline{
	San Jose, CA\ \
	831-295-1214\ \
}
\centerline{
	Email: \href{plguo002@gmail.com}{plguo002@gmail.com}\ \
	% \href{https://at15.github.io}{at15.github.io}\ \
	GitHub: \href{https://github.com/at15}{https://github.com/at15}\ \
	LinkedIn: \href{https://www.linkedin.com/in/at1510086}{https://linkedin.com/in/at1510086}}

\section{Work Experience}

\datedsubsection{\textbf{Google}: Software Engineer - EDA Farm}{Sunnyvale, CA \quad May 2019 - Present \quad}
\begin{itemize}
	\item Go and Python ?
\end{itemize}

\datedsubsection{\textbf{PayPal}: Software Engineer 2 - Multi cluster container orchestration platform in Go}{San Jose, CA \quad May 2018 - May 2019 \quad}
\begin{itemize}
	% \itemsep0em
	\item Working on internal \textbf{multi cluster} container orchestration platform using Apache \textbf{Mesos}, Aurora and Docker.
	% \item Built prototype of container storage on top of \textbf{Kubernetes} and Mesos for stateful service like database.
	\item Built REST API and cli tools in Go for collecting log and metrics along with \textbf{distributed tracing}.
	\item Introduced new deploy strategy and implemented dynamic internal system check during deploy.
	% \item Fixed race condition during deploy by merging user and internal events in state machine.
\end{itemize}

\datedsubsection{\textbf{PayPal}: Software Engineer Intern - Admin Server \& Dashboard in Go}{San Jose, CA \quad June 2017 - Sep. 2017}
\begin{itemize}
	% \itemsep0em
	\item Built API gateway with RBAC for internal container orchestration platform using Go.
	\item Enhanced dashboard using Angular 4, used by operation team to troubleshoot Java, Node app and manage cluster itself.
	% \item Introduced full text search using Apache Solr, wrote an open sourced Go client with cleaner API. % well, no longer maintained
	% \item Furthered continuous integration (CI) and deployment (CD) pipeline using Jenkins and Docker. % I hate jenkins in PayPal ...
\end{itemize}

% sorry GitCafe, the space on resume is limited
% \datedsubsection{GitCafe: Software Engineer Intern}{Shanghai, China \quad Jan. 2015 -- Mar. 2015}
% \begin{itemize}
% 	% \setlength\itemsep{0.1em}
% 	\item Reduced \textbf{Ruby on Rails} application load time by 5\% through optimizing regular expression in markdown parser.
% 	\item Fixed user subsubcription logic, solved 20\% pricing related user tickets.
% \end{itemize}

\datedsubsection{Dongyue Web Studio: (Part-time) Full stack web developer \& Tech lead}{Shanghai, China \quad Sep. 2013 -- Jan. 2016}
\begin{itemize}
	% \setlength\itemsep{0.1em}
	\item Led web and mobile team. Refactored online ticket booking application \href{https://tongqu.me}{tongqu.me} in 3 month, used by 10,000 students.
	\item Built high traffic website and REST API using PHP and MySQL, reduced database contention using job queue and cache.
	\item Utilized Redis as cache and rate limiter, increased QPS by 120\%, reduced database load by 40\%, filtered out most bot traffic.
	% \item Refactored jQuery codebase using AngularJS, increased homepage loading speed by 60\% using Ajax and pre-render.
\end{itemize}

\section{Project Experience}

\datedsubsection{\textbf{Go util library collection} \quad \href{https://github.com/dyweb/gommon}{github.com/dyweb/gommon}}{Open source side project \quad March. 2018 -- Present}

% \href{https://github.com/xephonhq/xephon-k}{https://github.com/xephonhq/xephon-k}

\begin{itemize}
	% \setlength\itemsep{0.1em}
	\item Built a high performance logging library with fine grained log level control across library dependencies.
	% \item Built error wrapping and inspection library based on go 2 draft proposal.
	% \item Implemented data structure missing from standard library, i.e. set, tree.
	\item Merged template rendering, static asset embedding into a single generator library as replacement for go:generate.
\end{itemize}

\datedsubsection{\textbf{Distributed database benchmark system} \quad \href{https://github.com/benchhub}{github.com/benchhub}}{UCSC \quad Nov. 2017 -- March. 2018}

% \href{https://github.com/xephonhq/xephon-k}{https://github.com/xephonhq/xephon-k}

\begin{itemize}
	% \setlength\itemsep{0.1em}
	\item Designed a specification for running distributed database benchmark.
	\item Built a queue based scheduler to run distributed database and workload generators.
	\item Stored benchmark results in time series databases and metadata in relation databases.
\end{itemize}

\datedsubsection{\textbf{Distributed Time Series Database} \quad \href{https://github.com/xephonhq/xephon-k}{github.com/xephonhq/xephon-k}}{UCSC \quad Nov. 2016 -- Present}

% \href{https://github.com/xephonhq/xephon-k}{https://github.com/xephonhq/xephon-k}

\begin{itemize}
	% \setlength\itemsep{0.1em}
	\item Implemented a distributed time series database on top of Cassandra in Go. Support both JSON and Protobuf via HTTP/2.
	\item Designed a columnar format modeled after Parquet and InfluxDB with higher compression and less write amplification.
	\item Created benchmark suite for Xephon-K, OpenTSDB, KariosDB, InfluxDB and a generic client for different TSDB.
	\item Surveyed popular TSDB design and implementation, made an interactive online report called \href{https://xephonhq.github.io/awesome-time-series-database}{awesome-time-series-database}.
\end{itemize}

\datedsubsection{\textbf{GPU accelerated in-memory time series processing} \quad \href{https://github.com/at15/ts-parallel}{github.com/at15/ts-parallel}}{UCSC \quad Apr. 2017 -- June 2017}

\begin{itemize}
	% \setlength\itemsep{0.1em}
	\item Expanded benchmark suite for different C++ GPU computing framework on CUDA and OpenCL, Thrust, Boost, ArrayFire.
	\item Implemented OLAP queries like top-K, group by for multi dimensional time series data on both CPU and GPU backends.
	      % \item Initiated a in memory column store with run length and dictionary encoding, saved 90\% space for regular time series data.
\end{itemize}

% \datedsubsection{\textbf{Distributed systems monitoring prototype}}{Shanghai Jiao Tong University \quad Mar. 2015 -- Jan. 2016}

% \begin{itemize}
% 	% \setlength\itemsep{0.1em}
% 	\item Enhanced monitoring system for distributed system using Cassandra and MongoDB written in Java and C++.
% 	\item Deployed in China Telecom's Kafka cluster, detected anomaly in disk and memory usage, improved capacity planning.
% \end{itemize}

% \datedsubsection{B+ index for \textbf{Hive}}{Shanghai Jiao Tong University \quad Nov. 2015 - Jan. 2016}

% \begin{itemize}
%   % \setlength\itemsep{0.1em}
%   \item Implemented B+ index for Hive, index is generated using MapReduce and stored in HDFS.
%   \item Accelerated point and range query using in memory LRU cache, supporting external cache like Memcached and Redis.
% \end{itemize}

% TODO: gommon

\section{Education}
% FIXED: too much space between educations
% FIXME: Package hyperref Warning: Token not allowed in a PDF string (Unicode) , caused by use \quad
\datedsubsection{MS. Computer Science \quad University of California Santa Cruz \quad \ GPA 3.9}{Sep. 2016 -- Mar. 2018}
% Courses: Distributed systems, Database Systems, Storage System
\datedsubsection{BS. \ Materials Science \quad \ Shanghai Jiao Tong University \qquad \  \quad GPA 3.3}{Sep. 2012 -- June 2016}

\section{Skills}
% https://tex.stackexchange.com/questions/16258/how-can-the-margins-around-a-table-set-to-0pt
\begin{flushleft}
	\begin{tabular}{@{}ll@{}}
		Language  & Go, JavaScript, Java, C++, Python, SQL, PHP, Shell                                                  \\
		Database  & Cassandra, MySQL, Elasticsearch, MongoDB, Redis, KairosDB, OpenTSDB, InfluxDB, Prometheus, Graphite \\
		DevOps    & Docker, Kubernetes, Mesos, Aurora                                                                   \\
		Framework & Vue, Angular, Laravel, Spring, Dropwizard, Express, Rails, CUDA, Hadoop
	\end{tabular}
\end{flushleft}

\end{document}
