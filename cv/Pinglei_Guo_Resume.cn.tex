\documentclass[11pt, letterpaper]{simple-cv}
\usepackage{enumitem}
% \setlist{nosep}

% NOTE: download and install font
% https://github.com/billryan/resume/tree/zh_CN/fonts
% https://github.com/adobe-fonts/source-han-serif/tree/release
% install SourceHanSerifSC-ExtraLight.otf
% TODO: use SourceHanSerif for all
\usepackage{xeCJK}
\setCJKmainfont[
BoldFont=Adobe Heiti Std,
ItalicFont=Adobe Kaiti Std,
SmallCapsFont=Adobe Heiti Std
]{Source Han Serif SC}
% Adobe Song Std

\begin{document}

\name{郭平雷}
% TODO: put into two lines, address + phone, and the rest
\centerline{
圣克鲁兹,加利福尼亚\ \
831-295-1214\ \
}
\centerline{
Email: \href{mailto:piguo@ucsc.edu}{piguo@ucsc.edu}\ \
% \href{https://at15.github.io}{at15.github.io}\ \
GitHub: \href{https://github.com/at15}{https://github.com/at15}\ \
Linkedin: \href{https://www.linkedin.com/in/at1510086}{https://linkedin.com/in/at1510086}}

\section{教育经历}
% FIXED: too much space between educations
% FIXME: Package hyperref Warning: Token not allowed in a PDF string (Unicode) , caused by use \quad
\datedsubsection{硕士 计算机科学\quad 加利福尼亚大学圣克鲁兹分校 \quad \ GPA 4.0}{2016 年 9 月 -- 2018 年 3 月}
% Courses: Distributed systems, Database Systems, Storage System
\datedsubsection{本科 材料科学 \quad 上海交通大学 \qquad  \qquad  \qquad \qquad \quad GPA 3.3}{2012 年 9 月 -- 2016 年 6 月}

\section{工作经历}

\datedsubsection{PayPal: 软件工程师实习}{圣何塞, 加利福尼亚 \quad 2017 年 6 月  - 2017 年 9 月}
\begin{itemize}
  \item 使用 \textbf{Go} 语言构建内部\textbf{容器编排平台} API 网关,已部署在生产环境的 \textbf{Mesos} 集群中。
  \item 使用 \textbf{Angular 4} 重构管理面板, 用于管理 Java, Node 应用和集群本身。
  \item 加入基于 \textbf{Solr} 的全文检索,开发并\textbf{开源}了性能更好以及支持 SolrCloud 的 Go 语言客户端。
  \item 完善了基于 \textbf{Jenkins} 和 \textbf{Docker} 的持续集成 (\textbf{CI}) 和持续交付 (\textbf{CD}) 工具流。
\end{itemize}

\datedsubsection{GitCafe: 软件工程师实习}{上海 \quad 2015 年 1 月 -- 2015 年 3 月}
\begin{itemize}
  \item 优化 \textbf{Ruby on Rails} 应用加载速度,改进 markdown 解析器以支持更多语法格式。修复用户付费订阅逻辑。
\end{itemize}

\datedsubsection{东岳网络工作室: 全栈工程师 \& 工作室负责人}{上海 \quad 2013 年 9 月 -- 2016 年 1 月}
\begin{itemize}
  \item 管理10人的网页端和移动应用开发团队,在3个月内重构已有2万用户的校内活动报名平台\href{https://tongqu.me}{同去网} \href{https://tongqu.me}{tongqu.me}。
  \item 使用 \textbf{Redis} 作为缓存和流量限制器,提高了 120\% 的响应速度,减少了 40\% 的 MySQL 负载,过滤机器人流量。
  \item 使用 \textbf{AngularJS} 重构了基于 \textbf{jQuery} 的代码,开发了静态资源管理和加载工具,提高了首页 60\% 的加载速度。
\end{itemize}

\section{项目经历}

\datedsubsection{\textbf{分布式时间序列数据库} \quad \href{https://github.com/xephonhq/xephon-k}{github.com/xephonhq/xephon-k}}{UCSC \quad 2016 年 11 月 -- 今}

% \href{https://github.com/xephonhq/xephon-k}{https://github.com/xephonhq/xephon-k}

\begin{itemize}
  \item 基于 \textbf{Cassandra} 使用 \textbf{Go} 语言实现, 支持 \textbf{JSON} 和 \textbf{Protobuf} 格式,使用 HTTP/2 协议。
  \item 设计了专用于时间序列数据存储的列存储格式,相对于 \textbf{Parquet} 和 \textbf{InfluxDB} 有更高的压缩率和更少的写放大。
  \item 创建了\textbf{压力测试集},支持 Xephon-K, OpenTSDB, KariosDB, InfluxDB,并开发了通用的时间序列数据库客户端。
\end{itemize}

\datedsubsection{GPU 加速的时间序列数据处理 \quad \href{https://github.com/at15/ts-parallel}{github.com/at15/ts-parallel}}{UCSC \quad 2017 年 4 月 -- 2017 年 6 月}

\begin{itemize}
  \item 扩展了 GPU 计算框架的\textbf{压力测试集},支持基于 CUDA 和 OpenCL 的 Thrust, Boost, ArrayFire。
  \item 在 CPU 和 GPU 上实现并对比了对多维时间序列数据的 OLAP 查询,如 top-K, group by。
  \item 实现了纯内存的列存储原型,使用行程压缩和字典压缩,对于规律的时间序列数据减少 90\% 的内存使用量。
\end{itemize}

\datedsubsection{分布式监控系统原型}{上海交通大学 \quad 2015 年 3 月-- 2016 年 1 月}

\begin{itemize}
  \item 基于 \textbf{Cassandra} 和 \textbf{MongoDB} 使用 \textbf{Java} and \textbf{C++} 实现,包含数据收集和异常检测的功能。
  \item 部署在中国电信的 \textbf{Kafka} 集群,检测到硬盘和内存使用\textbf{异常}。优化了\textbf{资源配置}和\textbf{预算管理}。
\end{itemize}

\section{技能}
\begin{tabular}{ l l }
 语言 &  Go, Java, JavaScript, TypeScript, PHP, Python, SQL, C++, Shell\\
 数据库 & Cassandra, Solr, MySQL, MongoDB, Redis, Elasticsearch, KairosDB, OpenTSDB, InfluxDB, Graphite\\
 运维 & Docker, Vagrant, Ansible, Mesos\\
 框架 & Angular, Laravel, Spring, Dropwizard, Express, Rails, CUDA
\end{tabular}

\end{document}
